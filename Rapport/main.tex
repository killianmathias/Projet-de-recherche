\documentclass{article}
\usepackage{graphicx}
\usepackage{url}

\title{Projet de recherche}
\author{Killian Mathias\\ Matthéo Girardclos}
\date{January 2025}

\begin{document}

\maketitle

\section{Introduction}

Créer des décors à la fois aléatoires et cohérent est un défi très compliqué. En effet, modéliser une montagne peut se réveler être un défi hardu, mais cela c'était avant de découvrir la génération procédurale de décors. Il existe de nombreux algorithmes permettant de générer ces décors de manière procédurale, le plus connu étant le bruit de perlin.

\subsection{Bruit de Perlin}

Le bruit de perlin, développé par Ken Perlin en 1985, est un système de génération pseudo-aléatoire. Son utilité principale

En informatique, la génération procédurale est la création de contenu numérique (en 2D, en 3D, des histoires, des dialogues, etc) en grande quantité de manière automatisée. La génération procédurale s'appuie sur des algorithmes pour créer.
En général, cette méthode est utilisé dans les domaines du Jeu Vidéo et du cinéma. Un exemple très connu de l'utilisation de la génération procédurale est le jeu Minecraft, qui créé un monde procédural.\\
\url{https://fr.wikipedia.org/wiki/Bruit_de_Perlin}\\
\url{https://fr.minecraft.wiki/w/Génération_du_monde}\\
\url{https://fr.wikipedia.org/wiki/Génération_procédurale}\\
\url{https://thebookofshaders.com/11/?lan=fr}
\end{document}
